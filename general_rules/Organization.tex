%%%%%%%%%%%%%%%%%%%%%%%%%%%%%%%%%%%%%%%%%%%%%%%%%%%%%%%%%
\section{Organization of the Competition}
\label{sec:procedure_during_competition}

\subsection{Stage System}\label{rule:stages}

The competition features a \iterm{stage system}. It is organized in two stages each consisting of a number of specific tasks. It ends with the \iterm{Finals}.

Each \iterm{stage} consists of a set of tasks grouped in two thematic scenarios.
% \iaterm{House Cleaner} and \iaterm{Party Host}.
The \iterm{Housekeeper} scenario features tasks related to cleaning, organizing, and giving maintenance; while the \iterm{Party Host} scenario focuses on attending guests' needs and providing general assistance during a party.

\begin{enumerate}
	\item \textbf{Robot Inspection:} For security, robots are inspected during setup days.
  A robot must pass \iterm{Robot Inspection} test (see~\refsec{sec:robot_inspection}) in order to compete.

	\item \textbf{Stage~I:} The first days of the competition are called \iterm{Stage~I}.
	All qualified teams can participate in \iterm{Stage~I}.
	The same task can be performed multiple times (See~\refsec{rule:score_system}).

	\item \textbf{Stage~II:} The best \emph{50\% of teams}\footnotemark (after Stage~I) advance to \iterm{Stage~II}.
	Here, tasks require more complex abilities or combinations of abilities.
	\footnotetext{If the total number of teams is less than 12, up to 6 teams may advance to Stage~II}
	
	\item \textbf{Final demonstration:} The best \emph{two teams} of each league, the ones with the highest score after Stage~II, advance to the final round.
	The final round features only a single task integrating all tested abilities.
	In order to participate in the Finals, a team must have solved at least one Stage~II task.
\end{enumerate}

In case of having no considerable score deviation between a team advancing to the next stage and a team dropping out, the TC may announce additional teams advancing to the next stage.


%%%%%%%%%%%%%%%%%%%%%%%%%%%%%%%%%%%%%%%%%%%%%%%%%%%%%%%%%
\subsection{Schedule}
\label{rule:schedule}

\begin{enumerate}
	\item \textbf{Thematic scenario blocks:} Each \iterm{thematic scenario} or \iterm{theme} is split in two \iterm{blocks}.
	At least two blocks are scheduled per day, each has a theme assigned and lasts no less than two hours.
	The \iaterm{Organizing Committee}{OC} announces the schedule during the setup days (see Table \ref{tbl:schedule}). One exception is the restaurant test which will have its own block for organizational reasons. The restaurant block will be part of the \iterm{Stage~II} party host block during which teams can either choose to participate in the restaurant test or not. The restaurant test can not be chosen in other time slots.

	\item \textbf{Slots:} The \iaterm{Organizing Committee}{OC} assigns at least two \iterm{test slots} of 5 minutes to each team in each block.
   The maximum number of \iterm{tests slots} will be announced during setup days by the \iaterm{Technical Committee}{TC} based on the available time and the number of participating teams.
	A team can solve any task during its test slot.
	Remaining block time can be used to assign additional testing slots to interested teams.
	Testing slots are randomly assigned to teams in each block.

	\item \textbf{Tests:} Teams must inform the OC in advance which task(s) they will try to solve.
	Only one task can be attempted per test slot.

	\item \textbf{Participation is default:} Teams have to indicate to the \iaterm{Organizing Committee}{OC} when they are \emph{skipping} a test slot. Without such indication, they may receive a penalty when not attending (see~\refsec{rule:not_attending}).
\end{enumerate}

% Please add the following required packages to your document preamble:
% \usepackage[table,xcdraw]{xcolor}
% If you use beamer only pass "xcolor=table" option, i.e. \documentclass[xcolor=table]{beamer}
\begin{table}[h]
	\centering\small
	\newcommand{\teams}[3]{%
		\tiny
		\begin{tabular}{c}%
			\textit{Slot 1, team $#1$}\\
			\textit{Slot 2, team $#2$}\\
			$\vdots$\\
			\textit{Slot $n/2$, team $#3$}\\
			\textit{Slot $n/2 + 1$, team $#1$}\\
			$\vdots$\\
			\textit{Slot $n$, team $#3$}\\
		\end{tabular}
	}
	\newcommand{\wcell}[2]{%
		\parbox[c]{2.5cm}{%
			\vspace{#1}%
			\centering%
			#2%
			\vspace{#1}%
		}%
	}
	\newcommand{\cell}[1]{\wcell{0.2\baselineskip}{#1}}


	\begin{tabular}{
		>{\centering\arraybackslash}m{2.5cm}|%
		>{\columncolor[HTML]{9AFF99}}c |%
		>{\columncolor[HTML]{9AFF99}}c |%
		>{\columncolor[HTML]{CBCEFB}}c |%
		>{\columncolor[HTML]{CBCEFB}}c |%
	}
	\multicolumn{1}{ c }{}
		& \multicolumn{1}{ c }{\cellcolor{white} Day 1 }
		& \multicolumn{1}{ c }{\cellcolor{white} Day 2 }
		& \multicolumn{1}{ c }{\cellcolor{white} Day 3 }
		& \multicolumn{1}{ c }{\cellcolor{white} Day 4 }
		\\\cline{2-5}
	\cell{Block 1\\\footnotesize(9:00 - 12:00)}
		& \cell{Housekeeper\\\teams{i}{j}{k}}
		& \cell{Party Host\\\teams{k}{i}{j}}
		& \cell{Housekeeper\\\teams{j}{i}{k}}
		& \cell{Party Host\\\teams{i}{k}{j}}\\\cline{2-5}

	\multicolumn{1}{ c }{}
		& \multicolumn{4}{ c }{\wcell{0.5\baselineskip}{\color{gray}Lunch}}\\\cline{2-5}

	\cell{Block 2\\\footnotesize(14:00 - 17:00)}
		& \cell{Housekeeper\\\teams{k}{j}{i}}
		& \cell{Party Host\\\teams{j}{k}{i}}
		& \cell{Party Host\\\teams{i}{j}{k}}
		& \cell{Housekeeper\\\teams{k}{i}{j}}\\\cline{2-5}

	\multicolumn{1}{ c }{}
		& \multicolumn{2}{ c }{\wcell{0.5\baselineskip}{\color[HTML]{029734}Stage 1}}
		& \multicolumn{2}{ c }{\wcell{0.5\baselineskip}{\color[HTML]{6668e5}Stage 2}}\\
	\end{tabular}

	\caption{Exemplary schedule:
		In this example there are two blocks for each theme in both stages.
		Each team has two test slots in every block meaning a team can choose to perform the same task four times, perform four different ones or anything in between for one theme in a stage.
	}
	\label{tbl:schedule}
\end{table}


\subsection{Score System}
\label{rule:score_system}
Each task has a main objective and a set of scoring bonuses.
To score in a test, a team must successfully accomplish the main objective of the task.
If a task's main goal has incremental scoring, the score for these increments is still awarded even if a team did not achieve the full goal. Bonus points are only awarded if at least one increment of the main goal is achieved.
Overall scoring is calculated as the sum of the maximum scores obtained in each individual test.

The \iterm{score system} has the following constrains
\begin{enumerate}
	\item \textbf{Stage~I:} The maximum score per task in \iterm{Stage~I} is \scoring{1000 points}.
	
	\item \textbf{Stage~II:} The maximum score per task in \iterm{Stage~I} is \scoring{2000 points}.

	\item \textbf{\iterm{Finals}:} Final score is normalized and a special evaluation is used.

	\item \textbf{Minimum score:} The minimum score per test in \iterm{Stage~I} and \iterm{Stage~II} is \scoring{0 points}.
	Teams cannot receive negative points.

	\item \textbf{Penalties:} An exception to \emph{minimum score} rule are penalties.
	Both penalties for not attending (see~\refsec{rule:not_attending}) and extraordinary penalties (see~\refsec{rule:extraordinary_penalties}) can cause a negative score.
\end{enumerate}




% Local Variables:
% TeX-master: "../Rulebook"
% End:
