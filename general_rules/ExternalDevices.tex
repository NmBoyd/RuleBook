% %% %%%%%%%%%%%%%%%%%%%%%%%%%%%%%%%%%%%%%%%%%%%%%%%%%%%%%%%%%
% 
% External Devices
% 
% %% %%%%%%%%%%%%%%%%%%%%%%%%%%%%%%%%%%%%%%%%%%%%%%%%%%%%%%%%%
\section{External Devices}
\label{rule:robot_external_devices}
Everything not part of the robot is considered an \iterm{external device}.
All external devices must be authorized by the \iaterm{Technical Committee}{TC} during the \iterm{Robot Inspection} test (see~\refsec{sec:robot_inspection}).
The \iaterm{Technical Committee}{TC} specifies whether an external device can be used freely, under referee supervision, and its impact on scoring.
In general, external devices must be removed quickly after the test.
	
\noindent \textbf{Remark:} The use of \iterm{wireless devices} is strictly prohibited. \iterm{External microphones}, hand microphones, and headsets are not allowed in OPL and it use is discouraged in DSPL and SSPL.

\subsection{On-site External Computing}
Computing resources that are not physically attached to the robot are considered \iterm{external computing resources}.
The use of up to 5 external computing resources is allowed, but only through the arena network (see \refsec{rule:scenario_wifi}) and with the previous approval of the \iaterm{Technical Committee}{TC}.

External Computing Devices must be placed in the \iaterm{\textbf{E}xternal \textbf{C}omputing \textbf{R}esource \textbf{A}rea}{ECRA} which is announced by the \iaterm{Technical Committee}{TC} during setup days.
A switch connected to the arena wireless network will be available to teams in the ECRA.
It is strictly forbidden to connect any kind of device or peripheral (e.g. screens, mouses, keyboards, etc.) to the computers in the ECRA during the competition.

A maximum of two laptops and two people from different teams are allowed at any time in the ECRA.
Teams using laptops as External Computing Devices must remove the device immediately after the test.
Once a test has started, all people must stay at least 1 meter from the ECRA.
Interacting with computers in the ECRA after the Referee has given the start signal will cause the immediate disqualification of the team in this test.

\noindent \textbf{Remark:} Robot operation must be able to operate safely when \iterm{external computing resources} are unavailable.



% On-line devices
\subsection{On-line External Computing}
\label{rule:robot_external_computing_online}
Robots are allowed to use \enquote{Cloud services}, \enquote{Internet API's}, and any other type of \iterm{external computing resource}.
Same restrictions for on-site external computing resources apply.

\noindent \textbf{Remark:} The competition organization doesn't guarantee or take any responsibility regarding the availability or reliability of neither the network nor Internet connection.
Teams' use of external computing resources is at their own risk.



% DSPL laptop
\subsection{Official Standard Laptop for DSPL}
\label{rule:osl_dspl}

In the Domestic Standard Platform League, teams may use the \iaterm{Official Standard Laptop}{OSL} connected to the Toyota HSR via Ethernet cable, safely located in the TOYOTA HSR \iterm{Mounting Bracket} provided by TOYOTA for this purpose.

\subsubsection{Technical Specifications}
The technical specifications for the Official Standard Laptop in the Domestic Standard Platform League are the following:


 \begin{itemize}
  \item \textbf{Brand and model:} DELL Alienware 15 or 17
  \item \textbf{CPU:} Core-i7 series
  \item \textbf{RAM:} 16GB or 32GB
  \item \textbf{GPU:} NVIDIA GeForce GTX 1070 or 1080
  \item \textbf{Storage:} Unrestricted.
\end{itemize}

No other brands or models will be accepted. There are no constrains regarding the software installed in the OSL but no additional hardware is allowed.

The referees, Technical Committee, and Organizing Committee members may run random checks anytime during the competition prior to the test to verify that the laptop in the TOYOTA HSR \iterm{Mounting Bracket} has no additional hardware plugged in, and matches the authorized specifications.


% Local Variables:
% TeX-master: "../Rulebook"
% End:
