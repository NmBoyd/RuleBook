%%%%%%%%%%%%%%%%%%%%%%%%%%%%%%%%%%%%%%%%%%%%%%%%%%%%%%%%%
\section{Team Registration and Qualification}


\subsection{Registration and Qualification Process}
\label{rule:participation}

Each year there are four phases in the process toward participation:
\begin{enumerate}
	\item \iterm{Intention of Participation} (optional)
	\item \iterm{Preregistration}
	\item \iterm{Qualification} announcements
	\item Final \iterm{Registration} for qualified teams
\end{enumerate}
Positions 1 and 2 will be announced by a call on the \iterm{RoboCup@Home mailing list}. Preregistration requires a \iterm{team description paper}, a \Term{video}{qualification video} and a \Term{website}{Team Website}.

\subsection{Qualification Video}
As a proof of running hardware, each team has to provide a \iterm{qualification video} showing at least two from the following abilities (minimum requirement):
\begin{itemize}
	\item Human-Robot interaction
	\item Navigation (safe, indoors with obstacle avoidance).
	\item Object detection \& manipulation.
	\item People detection
	\item Speech recognition.
	\item Speech synthesis (clear and loud).
\end{itemize}

Showing some of the following abilities is recommended:
\begin{itemize}
	\item Activity recognition
	\item Complex speech recognition
	\item Complex action planning
	\item Gesture recognition
\end{itemize}


Videos should be self-explanatory and designed for a general audience, showing the  robot solving complex tasks. The minimum to qualify requires proving the robot is able to solve successfully at least one test of the current or last year's rulebook. For robots moving slowly, we suggest to speed-up videos. When doing so, please specify the speed factor being used (e.g.~2x, 5X, 10X). The same applies for slow motion scenes. Videos should not exceed the average time for a test (max.~\SI{10}{\minute}).

\subsection{Team Website}

The \iterm{Team Website} should be designed for a broader audience, and include scientific material (scientific papers, datasets, and documented open source code). Requirements are as follows:

\begin{enumerate}

	\item \textbf{Multimedia:}~As many photos and videos of the robot(s) as possible.

	\item \textbf{Language:}~The team website has to be in English. Other languages may be also available, but English must be default language.

	\item \textbf{Team:}~Comprehensive list of the team members including brief profiles.

	\item \textbf{RoboCup:}~Link to the league website and previous participation of the team in RoboCup.

	\item \textbf{Scientific approach:}~Include research lines, description of the approaches, and information on scientific achievements.

	\item \textbf{Publications:}~Relevant \iterm{publications} from 5 years up to date. Downloadable publications are scored higher during the qualification process.

	\item \textbf{Open source material:}~Blueprints, datasets, repositories or any kind of contribution to the league is highly scored during qualification process.
\end{enumerate}


\subsection{Team Description Paper}
\label{rule:website_tdp}
The \iaterm{team description paper}{TDP} is a scientific paper which must contain a description of your main research, including the scientific contribution, goals, scope, and results.

~\\\noindent Preferably, it should also contain the following:
\begin{itemize}
	\item The focus of research and the contributions in the respective fields
	\item Innovative technology (if any)
	\item Re-usability of the system for other research groups
	\item Applicability of the robot in the real world
	\item Photo(s) of the robot(s)
\end{itemize}

~\\\noindent As addendum in the 9th page (after references) please include:
\begin{itemize}
	\item Team name
	\item Contact information
	\item Website url
	\item Team members' names
	\item Photo(s) of the robot(s), unless included before.
	\item Description of the hardware used
	\item Brief, compact list of \iterm{external devices} (See~\refsec{rule:robot_external_devices}), if any.
	\item Brief, compact list of 3rd party reused software packages (e.g.~ROS' \texttt{object\_recognition} should be listed, but not OpenCV).
	\item \textbf{[Open Platform League only]} Brief description of the hardware used by the robot(s).
\end{itemize}

~\\\noindent The TDP has to be in English, up to eight pages in length and formatted according to the guidelines of the RoboCup International Symposium without altering margins or spacing. It goes into detail about the technical and scientific approach.

Please notice that, during qualification process, TDP will be scored by its scientific value, novelty and contributions.


%% %%%%%%%%%%%%%%%%%%%%%%%%%%%%%%%%%%%%%%%
\subsection{Qualification}
\label{rule:qualification}

During the \iterm{qualification process} a selection will be made by the \iaterm{Organizing Committee}{OC}. Taken into account and evaluated in this decision process are:
\begin{itemize}
	\item Content on the team website, publications and open source resources are most important
	\item Number of abilities shown in the qualification video
	\item Complexity of the tasks shown in the qualification video
	\item Scientific value, novelty and contributions in the \iterm{team description paper}. %, and
	% \item the information in the \iterm{RoboCup\char64Home Wiki} (added by the team).
\end{itemize}
(Additional) evaluation criteria are:
\begin{itemize}
	\item Performance in previous competitions
	\item Relevant scientific contributions and publications
	\item Contributions to the RoboCup@Home league
\end{itemize}

\paragraph{Important note to Standard Platform Leagues:} Only unmodified robots may compete in Standard Platform Leagues. Any \textit{slight} modification made to the robot found in the qualification material will automatically disqualify the team and registration to the international competition will not be possible  (See~\refsec{rule:spl-mods}).

% For getting considered in the evaluation, be sure to insert your team's name when adding information to the \iterm{RoboCup\char64Home Wiki}.


% Local Variables:
% TeX-master: "../Rulebook"
% End:
